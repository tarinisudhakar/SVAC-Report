\documentclass[a4paper, 12pt, twoside]{article}
\usepackage[english]{babel}
%Margins
\usepackage[left=25.4mm, right = 25.4mm, top=25.4mm, bottom=25.4mm, includefoot]{geometry}

%For beautiful paragraph spacing
\usepackage{parskip}
\setlength{\parindent}{0in}
\usepackage{enumitem}
\usepackage{setspace}

%Adding Pictures
\usepackage{graphicx}
\usepackage{float}
\usepackage{pdflscape} %converts page from to portrait to landscape mode
\usepackage{wrapfig}
%Header and Footers
\usepackage{fancyhdr}
\pagestyle{fancy}
\fancyhead{}
\fancyfoot{}
\fancyfoot[RO]{Centre for Civil Society \hspace{1mm} \textbar \hspace{1mm} \thepage\ }
\fancyfoot[LE]{ \thepage \hspace{1mm} \textbar \hspace{1mm}SVAC Report 2018 }
\renewcommand{\headrulewidth}{0pt} %change the pt width to insert header line
\renewcommand{\footrulewidth}{0pt} %change the pt width to insert footer line

%Maths
\usepackage{amsmath}
\usepackage{mathtools}
%Tables
\usepackage{booktabs}
\usepackage{multicol}
\usepackage{subfig}
\captionsetup{aboveskip=14pt,}
\captionsetup[table]{singlelinecheck = false}
\usepackage{rotating}
\usepackage{longtable}
\usepackage{makecell}
\renewcommand\theadfont{\scriptsize}
\usepackage{array}
\usepackage{cellspace}
\setlength\cellspacetoplimit{3pt}
\setlength\cellspacebottomlimit{3pt}

%Coloured Boxes
\usepackage{xcolor}
\usepackage{mdframed}
%Custom Spacing
\usepackage{setspace}
\pretolerance=10000
\emergencystretch=10pt

%preventing widow/orphan lines
\widowpenalty10000 % required to prevent widow/orphan lines
\clubpenalty10000 % required to prevent widow/orphan lines
%Defining Colours
\definecolor{CCSbrown}{RGB}{163, 86, 37}
\definecolor{CCSgrey}{RGB}{105, 105, 105}
\definecolor{SVACgreen1}{RGB}{106, 168, 79}
\definecolor{SVACgreen2}{RGB}{217, 234, 211}
\definecolor{SVACgreen3}{RGB}{182, 215, 168}
\definecolor{SVACyellow1}{RGB}{255, 217, 102}
\definecolor{SVACyellow2}{RGB}{255, 242, 204}
\definecolor{SVACred1}{RGB}{224, 102, 102}
\definecolor{SVACred2}{RGB}{234, 153, 153}
\definecolor{SVACred3}{RGB}{221, 126, 107}
\usepackage{sectsty}
\usepackage{titlesec}
\chapterfont{\color{blue}}  %sets colour of chapters font
\sectionfont{\color{CCSbrown}}  %sets colour of section font
\subsectionfont{\color{CCSblack}} %sets colour of subsection font
\subsubsectionfont{\color{black}} %sets colour of subsubsection font
%Bibliography
\usepackage[authordate, backend=biber]{biblatex-chicago}
\addbibresource{SVAC.bib}
\hypersetup{
colorlinks,
linkcolor = CCSblack,
citecolor = CCSbrown,
urlcolor = CCSblack}
\usepackage{blindtext}

%Symbols
\newcommand{\quotes}[1]{``#1''}


\begin{document}

\newpage

\begin{titlepage}
\begin{center}
\line(1,0){300}\\
[0.25in]
\huge{\bfseries \textcolor{CCSbrown} {Annual Report on Implementing the Street Vendors Act 2014}} \\
[0.5cm]
%\large{Subtitle} \\
\line(1,0){200}\\
[2in]
\includegraphics[width = 6cm]{CCSlogo.jpg} \\
[1.5cm]
[1.5cm]
%{\Large Centre for Civil Society} \\
{\Large New Delhi, India} \\
{\Large January 2019} \\
[1.85cm]


\end{center}
\end{titlepage}

%======================ACKNOWLEDGEMENTS==========================
\section*{Acknowledgements}


The analysis would not have been possible without the cooperation from the survey respondents including officials of Municipal Corporation of Gurugram (MCG) and New Delhi Municipal Council (NDMC), TVC members (including NGOs, private enterprises, vendor associations) and the vendors of Gurugram and Delhi.


The research, drafting, and publication of this report was carried out by Centre for Civil Society’s in-house team comprising Bhuvana Anand, Jayana Bedi, Pooloma Ghosh, Ritika Shah, Shruti Gupta and Vidushi Sabharwal. We are also grateful to Prashant Narang for leading the authorship on legal analysis. We thank C. K. Pushyami, Tarini Sudhakar, and Usha Sondhi Kundu for designing and Bushra Rashid for proofreading the study.


\newpage

\newpage
\newlist{abbrv}{itemize}{1}
\setlist[abbrv,1]{label=,labelwidth=1in,align=parleft,itemsep=0.1\baselineskip,leftmargin=!}

\section*{List of Abbreviations}
\addcontentsline{toc}{section}{List of Abbreviations}
\begin{abbrv}
      	\item[CBO]			Community-based Organisation
	\item[CEO]			Chief Executing Officer
        	\item[CP]				Connaught Place
        	\item[DAY-NULM]		Deendayal Antyodaya-National Urban Livelihood Mission
        	\item[DCP]			Deputy Commissioner of Police
        	\item[DDA]			Delhi Development Authority
        	\item[EDMC]			East Delhi Municipal Corporation
        	\item[FSSAI]			Food Safety and Security Authority of India
	\item[GNCTD]			Government of National Capital Territory of Delhi
	\item[HQ]				Head Quarters
	\item[HUDA]			Haryana Urban Development Authority
	\item[IGSSS]			Indo-Global Social Service Society
	\item[LOI]				Letter of Interest
	\item[MCD]			Municipal Council of Delhi
	\item[MCG]			Municipal Corporation of Gurgaon
	\item[MoHUA]			Ministry of Housing and Urban Affairs
	\item[NASVI]			National Association for Street Vendors India
	\item[NCR]			National Capital Region
	\item[NDMC]			New Delhi Municipal Council
	\item[NGO]			Non-Government Organisation
	\item[NoDMC]			North Delhi Municipal Corporation
	\item[OECD]			Organisation of Economic Co-operation and Development
	\item[ORF]			Observer Research Foundation
	\item[PWD]			Public Works Department
	\item[RTI]				Right To Information
	\item[RWA]			Residents Welfare Association
	\item[SDMA]			South Delhi Municipal Corporation
	\item[SEWA]			Self-Employed Women's Association
	\item[SSSPL]			Spick and Span Services Pvt Ltd
	\item[SUDA-H]			State Urban Development Authority, Haryana
	\item[SUSV]			Support to Urban Street Vendors
	\item[TVC]			Town Vending Committee
	\item[UD]				Urban Development
	\item[ULB]				Urban Local Body
	\item[UT]				Union Territory
	\item[WIEGO]			Women in Informal Employment: Globalizing and Organizing
\end{abbrv}

%===================EXECUTIVE SUMMARY==============================
\newpage
\section*{Executive Summary}
\addcontentsline{toc}{section}{Executive Summary}


Centre for Civil Society advances economic and property rights of informal workers who are often at risk of abuse from authorities in absence of clear laws or bounds on state power. This report looks at the most visible form of informal employment in urban Indian cities: street vendors. It evaluates the progress in institutionalising mechanisms to protect and regulate vending since enactment of Street Vendors (Protection of Livelihood and Regulation of Street Vending) Act, 2014 (henceforth referred to as “the Act”).


The central policy problem is to manage such conflicting and competing interests of vendors, pavement users, local residents, vehicular traffic, and urban space managers over use of public space. Against this background of competition for public space, this report studies the use of new legal tools to resolve conflicts by the higher courts and evaluates the distance covered by states in implementing the Act. There are three parts to this report: analysis of judgments; a cross-state index tracking extent and depth of implementation; and case studies diving deeper into the functioning of two TVCs in two cities—Delhi and Gurugram.

Building on our work last year, we analysed judgements passed between January 2017 and September 2018 on disputes relating to the Act,  for developing a comprehensive understanding of the jurisprudence. The most contested issues continue to centre around the eviction of street vendors. Repeated challenges in court arise from the ambiguous legal definition of street vendors, the judiciary's conflicted understanding of no-vending zones, and varying interpretations of the overriding effect of the Act. Far from fulfilling the intended legislative objectives, we find that the judgements passed by various High Courts fail to establish necessary checks on the actions of municipal authorities, and penalise the very street vendors that the Act seeks to protect.

In 2017, we developed a cross-state index and tracked the progress of individual states in implementing the Act. In 2018, we rechecked the status of implementation. We measured absolute and relative progress in implementing the act by studying the data provided by the Ministry of Urban Housing and Affairs (MoUHA) and state governments. Five years since enactment of the Act, we find several areas of disregard for implementing the Act and a sluggish pace of progress in implementation. The Act states that there should be atleast one TVC in each Urban Local Body (ULB). However, there are only 2,382 TVCs for 7,263 ULBs in India. 42\% of these TVCs do not have vendor representatives, defeating the purpose of a “participatory committee.” Only four out 28 states and 2 union territories that responded have a grievance redressal committee.

To measure relative progress in implementing the Act, we give each state a score based on the depth of implementation. The index captures performance on all 11 steps encompassing publication and notification of schemes, enumeration and registration of vendors, setting up TVCs with vendor representation, drawing plans to demarcate vending zones and institutionalising the grievance redressal committee among others. Mizoram, Chandigarh and Rajasthan are the top three state. Two of these states had a state Act on vending based on the model bill provided by the central government. Though the state Acts were repealed, all orders and actions taken under them were deemed to be issued under the provisions of the Central Act and continue to legally tenable. West Bengal and Nagaland are the worst performing states. West Bengal has 3 TVCs for 239 towns and Nagaland has 2 TVCs for 11 towns. Both states have only implemented two out of 11 steps of implemented.




Overall, we find that despite the establishment of a comprehensive legal framework in the Act, implementation by the state has been sluggish. Even though the Act prioritises vendor protection and has an overriding effect, the court has contravened the spirit of the Act and reinforced pre-existing national, state and local laws. We are yet to see if the new democratic and vendor-led governance, when implemented, will lead to new ways of thinking about place for vendors in Indian cities.


%===================STREET VENDORS IN INDIA: AN OVERVIEW=============
\newpage
\section*{Street Vendors in India: An Overview}
\addcontentsline{toc}{section}{Street Vendors in India: An Overview}

Despite their contribution to the urban economy, vendors are often considered “anti-social, anti-developmental, dirty, unaesthetic and unhygienic” (WIEGO 2014). They are frequently targeted, harassed and evicted by government officials. In most cases of eviction, they reappear in connivance with municipal and police authorities. Even the Supreme Court has taken note of how vendors are a “harassed lot and are constantly victimized by the officials of the local authorities, the police, etc.” (Maharashtra Ekta Hawkers Union v. State of Maharashtra(1 November 2017) –[MEHU]).


\subsection*{What Rights Do Street Vendors Have?}
\addcontentsline{toc}{subsection}{What Rights Do Street Vendors Have?}

Two kinds of rights define the ownership of a vendor over a property. The first is the “right to vend from a particular area” or rights over the immovable property hawkers operate from. The second is the “right to ownership of the movables that are used for conducting trade” \parencite{ccspaper}.

The salient policy problem, given the contrasting and often competing interests over the use of public space, is marking the precedence, extent and limits of the rights of all users. Managing conflicting claims of street vendors, pavement users, local residents, vehicular traffic and urban space managers is central to designing and implementing the regulatory framework for street vending.
\newpage
\subsection*{How Does the Street Vendors (Protection of Livelihood and Regulation of Street Vending) Act, 2014, Apply These Rights?}
\addcontentsline{toc}{subsection}{How Does the Street Vendors (Protection of Livelihood and Regulation of Street Vending) Act, 2014, Apply These Rights?}

\subsection*{Explicit Recognition of Vending as a Legitimate Livelihood}

The central government enacted the Street Vendors Act, 2014, which is legally binding on all state and local governments. Prior to this, the National Policy on Urban Street Vendors 2004 was the guiding document to tackle the issues of vendors’ rights. However, the policy was only a set of guidelines and states were not legally bound to enforce it. The Act seeks to “protect the rights of urban street vendors and regulate street vending activities and matters connected therewith or incidental thereto.”

\subsubsection*{Participation of Civil Society in Spatial Regulation and Institutional Mechanisms to Protect Vendors}



\textbf{Channel for Negotiation Between Civil Society and the State}

The Act relegates powers to enumerate, identify and allocate vendors to zones to a TVC, allowing for decentralised governance. It redistributes powers exclusively held by municipal bodies and the police between street vendors, market associations and local residential association. To protect vendors’ rights, the Act requires at least 40\% of the TVC members to be vendors. An additional 10\% of the members must be from nongovernment and community-based organisations. This way the Act creates mechanisms for participatory decision making, reducing the scope for exclusionary practices such as harassment or relocation to noncommercial locations.

\textbf{Enumerate and Certify All Without Caps on the Number of Licences}

A TVC is required to conduct periodic enumeration (once in every 5 years) of local street vendors. There is no artificial limit of the number of street vendors, creating a permitting framework instead of a licensing regime. The TVC is required to accommodate all identified vendors in the vending zones, subject to the holding capacity.  If the number of vendors identified exceeds the holding capacity, they are to be accommodated in an adjoining location to “avoid relocation.” The Act categorically rules out any attempt to evict or relocate vendors and declare no-vending zones, prior to the completion of the survey.

\newpage
\textbf{Independent Dispute Redressal Mechanism to Ensure Grievances Are Heard }

The Act mandates the state government to establish one or more independent committees, chaired by ex-judicial officers, for redressal of disputes and complaints filed by the street vendors. The Act categorically excludes any “employee of the appropriate government or the local authority” from the committee to ensure impartial decisions. In doing so, the Act recognises the need to place checks on administrative decisions and curb harassment by municipal authorities.

\subsection*{But, How Good Is the Implementation?}

In 2017, we developed an index to rank states on their progress in and fidelity to the implementation of the Act. We lodged applications under the Right to Information Act, 2005 across 30 states and union territories in India, compiled court judgements and analysed secondary sources including news stories. After collating data, we used a weighted scoring method to assess state-wise performance.


The report evaluates the extent of application of the new legal and governance framework to spatial conflicts and its impact on vendors. We argue that the ineffective role played by the judiciary and lackadaisical implementation by the states, makes the Act fall short of fulfilling its intended objective, that is protecting the rights of urban street vendors.
%===================HOW HAS THE JUDICIARY INTERPRETED===============
\section*{How Has the Judiciary Interpreted the Act?}
\addcontentsline{toc}{section}{How Has the Judiciary Interpreted the Act?}


\subsection*{Summary of Issues Contested and Case Outcomes}
\addcontentsline{toc}{subsection}{Summary of Issues Contested and Case Outcomes}

\begin{itemize}[nosep]
\item \textbf{Eviction is the single most contested issue:} In 47 out of 57 cases, vendors or their representatives challenged what they considered “unlawful eviction.”

\begin{itemize}
\item Eviction in these cases resulted mainly from: (1) the ambiguity surrounding the definition of a vendor under the law and their identification in the absence of enumeration surveys; and (2) the conflicted understanding of vending and no-vending zones, with some courts still upholding the demarcations made before the Act.

\item Other issues of contention relate to the representation of vendors in TVC, transparency in street vendor elections, enumeration of street vendors, permissions to change trade and implementation of the Act.

\end{itemize}
\end{itemize}

\begin{itemize}

\begin{itemize}
\item Of the 24 cases where courts ruled against the vendor petitioner, 20 petitions challenged unlawful evictions or harassment. 14 out of these 20 adverse decisions were pronounced by the High Court of Delhi.
\item In 21 cases where courts issued deferrals, the decisions likely cemented the status quo.
\end{itemize}
\end{itemize}

\subsection*{Are Vendor Evictions Lawful?}
\addcontentsline{toc}{subsection}{Are Vendor Evictions Lawful?}

Between January 2017 and September 2018, 47 petitioners (including vendors and vendor associations) challenged what they considered unlawful eviction. Petitioners demanded protection under Section 3(3) of the Act that requires local TVCs to survey all existing vendors and issue identity cards before conducting any evictions. Despite the provision, vendors continue to be evicted due to various reasons, especially in areas where TVCs are not functional yet. Such evictions retain historical biases against vendors. These evictions contravene the Act, which explicitly charges state machinery to take comprehensive measures to check and control the practice of forced evictions. As a result of varying judicial interpretations of the definition of a street vendor, the legal status of vendors in many places remains unclear.

\subsubsection*{Evictions Based on Exclusionary Definitions of a Street Vendor}


Two high courts—the High Court of Delhi and the High Court of Himachal Pradesh—have added new criteria to determine who is a vendor and excluded many from the ambit of legal protection under the Act. In contrast, the High Court of Kerala adopted a more progressive approach, arguing that Act extends protection to all vendors, irrespective of their current legal status.

\textbf{High Court of Delhi: Only Protects Vendors in Official but Outdated Lists}



In Bhikki Ram vs New Delhi Municipal Council 2017, given the “over-crowded area,” the NDMC allowed only those vendors to vend “whose names find mention in the list of 628 eligible persons prepared by the NDMC or are licensed vendors or hawkers.” The Court chose not to interfere in the policy decision of the NDMC as long as the municipal corporation acted in a “fair, just and uniform manner without any favour of any kind.”



\textbf{High Court of Himachal Pradesh: Adds Another Caveat to the Definition of a Vendor}


\textit{26. It is also not that every hawker has got a right of protection from ejectment/eviction, under the provisions of the Act ... There is no automatic application of the Act qua every vendor, who under misconception chooses to sit on any place or time on a public property, vending anything and everything. Persons, who come to the State, seeking employment, only on weekends or during tourist season, when tourists throng the State in large numbers, have no right of protection under the Act.}


\textbf{High Court of Kerala: Set Exemplary Inclusive Parameters for Defining “Street Vendor”}

The High Court of Kerala argues that the Act extends protection to all vendors irrespective of their legal status. In 2014, it disallowed evictions and allowed the petitioner to continue vending until the procedures laid out in the Act were implemented (Thankappan vs The District Collector 2014). As opposed to the narrow interpretation of the Act adopted by the High Court of Delhi and Himachal Pradesh in 2017 to 2018, the High Court of Kerala sets a desirable example of who ought to be considered a street vendor.

\subsubsection*{Evictions Based on Pre-2014 Demarcations of No-vending Zone}

The Act prohibits municipal and local authorities from declaring no-vending zones until all vendors are listed and a TVC with vendor representation is established. Despite this, the High Court of Delhi allowed the municipal corporations to continue “regular eviction drives.” In seven cases, the Court upheld the pre-2014 demarcation by municipal bodies as an additional ground to justify eviction.  Although the Court specified that such no-vending zones remain valid only until the TVCs start functioning, it allowed interim eviction. This interpretation contravenes Clause 3 of the First Schedule of the Act, which states that no-vending zone should not be declared before surveys are completed and plans formulated.



\subsubsection*{Evictions Based on Balancing “Public Interest”}

Sections 3(13) and 4(18) of the Act mandate local authorities to relocate vendors removed from a particular area due to any public purpose in consultation with the TVC. Judgements passed by the High Court of Delhi and the High Court of Calcutta, however, allow for evictions based on promoting public interest without any consideration for relocation.

In 2016, the High Court of Delhi allowed the civic agencies to evict vendors from no-vending zones declared prior to the enforcement of the Act to balance “public interest” (WPC 6130/2016 vide order dated 05.10.2016). In 2017, the Court argued that “Being pitched between the conflicting rights of the livelihood of the street vendors versus the life and security of the public in general, including the street vendors... we are of the opinion that the former must bow to the latter as without life and security, no question of earning a livelihood can arise” (Vyapari Kalyan Mandal Main Pushpa v. South Delhi Municipal Corporation, 2017).


\subsubsection*{Evictions Based on Misinterpretation of “Overriding Effect” Clause}

Section 33 of the Act 2014 gives it an overriding effect over all other laws, whether local or state, in case of inconsistencies. This prevents harassment of street vendors on the basis of other statutes, bye-laws or executive orders. Different high courts have, however, interpreted the provision in ways that still uphold and give precedence to state and municipal laws.

\textbf{High Court of Kerala: Overriding Effect Only Applicable in Case of Inconsistency}



\textbf{High Court of Madras: State Municipal Law Is Not Overwritten by the Act}

The High Court of Madras held that the state municipal law—Tamil Nadu District Municipalities Act, 1920—is not overwritten by the Street Vendors Act, 2014 (T. Ramalingam and Ors v. The Secretary to the Government and Ors, 2018). The Court allowed eviction as the Street Vendors Act, 2014, is only “to protect the livelihood rights of the street vendors and to regulate their street vending activities in urban areas across the country and for matters connected therewith. Therefore, it is lucidly clear that the ingredients of Tamil Nadu District Municipalities Act, 1920, especially Sections 180(A) and 182 of the Act  are mandatory... and in fact, the Street Vendors Act, 2014, is regulatory in character.” By laying emphasis on the “mandatory” clauses of the Tamil Nadu District Municipalities Act, 1920, the Court gives it greater precedence over and above the Street Vendors Act, 2014.

\textbf{High Court of Gujarat: No Protection to Vendors That Violate State Laws}

The High Court of Gujarat did not protect petitioners who operate “illegal and unauthorised Pucca constructions” on a public street as it violates other state laws (Vakatar Samatbhai Ghusabhai v. State of Gujarat, 2018; Vagha Bachu Changa v. State of Gujarat and Ors, 2017).

\subsubsection*{Eviction Decisions Deferred to TVCs}


In Vijay Kumar Sahu and Ors v. Govt. of NCT and Ors (2018), the petitioner-vendor contended that he had been vending at the same place for the past 21 years and was now being harassed. The area in question was not a no-vending zone, and hence the High Court of Delhi had no reason, even by its own yardstick, to deny the protection to the petitioner under Section 3(3) of the Act. Despite this, the Court deferred the decision and asked the petitioner to “approach the TVC as and when it is functional, with all the supporting documents,” without granting any interim protection.


When faced with similar questions, the High Court of Madras and the High Court of Kerala referred the questions of fact to the municipal agencies or the TVCs. The courts asked the municipal agency or TVC (if constituted) to consider the application filed by the petitioner-vendor within the stipulated time and decide the matter giving a reasoned order.

\subsection*{Vendors Lacking Recourse to Challenge Voter Lists for TVC Elections}
\addcontentsline{toc}{subsection}{Vendors Lacking Recourse to Challenge Voter Lists for TVC Elections}



A representative and well-functioning TVC is an essential feature of the Act. Denying vendors from raising objections against the list produced by the DMC may result in limited participation and vendor representation in TVCs. It hinders the attempt of the Act to undo the historical biases that have plagued decisions around the number of vendors and vending zones.


%===================HOW DO STATES FARE ON IMPLEMENTATION===========
\section*{How Do States Fare on Implementation?}
\addcontentsline{toc}{section}{How Do States Fare on Implementation?}
	Government officials—responsible for urban planning, controlling congestion and maintaining hygiene—are constantly faced with the need to “do something about street vending” \parencite{bromleypaper}. Vendors are a low priority for state governments, and governance and management of vendors is left to those at the bottom of the administrative hierarchy, including police inspectors and officials of the municipal corporation.


	The Act is an attempt to systematically fill the regulatory lacuna, regularise street vendors, open channels for negotiation between stakeholders and minimise extractive opportunities. The Act identifies general principles and lays out restrictions on what can and cannot be done and leaves the application of these principles to Urban Local Bodies (ULBs) given that regulating and managing vending requires localised solutions and consensus. But is it working?

	We identified 11 distinct steps required of state governments under the Act and gathered data to determine how far states have progressed (Table \ref{tab: SVACsteps}). We filed Right to Information (RTI) applications in 28 states and 7 Union Territories (UTs) (Appendix \ref{sec: Appendix 1}) asking 11 questions (one corresponding to each step) to help substantiate state progress. For example, the Act requires a state to notify the rules for implementing the Act. Correspondingly, we asked if the state government had drafted and notified the rules. Where possible, we asked for quantitative data to determine the extent of implementation. For example, on the exercise of the survey and issuance of identity cards, we asked how many vendors were identified and how many were issued identification.
%Table 1: Steps to Implement the Street Vendors Act 2014
\begin{table}[htpb]
\caption{Steps to Implement the Street Vendors Act 2014}
\label{tab: SVACsteps}
\begin{tabular}{ l  l } %Alignment of Text
\toprule
Step 1	&	State government to draft and notify the \textbf{rules} for implementing the Act\\
Step 2 	&	State government to draft and notify the \textbf{scheme} for implementing the Act\\
Step 3	&	\textbf{State government to form the Grievance Redressal Committee}\\
Step 4	&	\textbf{State government to form the TVC}\\
Step 5	&	\textbf{Election for vendor representation in the TVC}\\
Step 6	&	\textbf{TVC to conduct a survey of vendors}\\
Step 7	&	TVC to \textbf{issue ID cards to vendors}\\
Step 8	&	\textbf{TVC to earmark vending zones}\\
Step 9 	&	\textbf{Local authority to draft and publish a street vending plan}\\
Step 10	&	\textbf{TVC to publish the street vendor charter}\\
Step 11	&	Local authority to \textbf{assign office space to the TVC}\\
\bottomrule
\end{tabular}
\end{table}

These 11 questions were also sent to the Deendayal Antyodaya–National Urban Livelihood Mission (DAY-NULM) under the Ministry of Housing and Urban Affairs. NULM sets out the strategy and operational guidelines for implementing the Act under its Support to Urban Street Vendors (SUSV) component. NULM coordinated with the state representatives multiple times and shared updated data at different stages of the report. The data was verified and updated last in January 2019.

	We received a response from 28 states and 2 UTS.\footnote{ We have referred to 28 states and 2 UTs as 30 states throughout the text.} 5 UTs—Andaman and Nicobar Island, Dadra and Nagar Haveli, Daman and Diu, Delhi and Lakshadweep—did not respond to the RTI/NULM request for data. The Act does not apply to Jammu and Kashmir.

	In this section, we discuss our findings on the status of compliance of states with the Act, 4 years from its onset.

\subsection*{Summary of Findings}
\addcontentsline{toc}{subsection}{Summary of Findings}

\begin{itemize}
	\item Four states\textemdash Arunachal Pradesh, Karnataka, Telangana and Nagaland\textemdash are yet to comply with step 1 on notifying the rules.
	\item Nineteen states have completed step 2, notification of scheme but but well after the statutory deadline of October 2014.
	\item Four states---Assam, Madhya Pradesh, Punjab and Uttarakhand --- have implemented step 3, the formation of the Grievance Redressal Committees.
	\item Of the 7,263 towns from 30 states, 33\% have complied with step 4, formation of TVC.
	\item Of the 2,382 TVCs formed, 58\% have complied with step 5, vendor representation in the TVCs.
	\item Of the TVCs formed, 98\% have completed step 6, vendor enumeration.
	\item Of the TVCs formed, 50\% have issued identity cards to 75\% of the identified vendors.
	\item Twenty percent of the TVCs complied with step 8, have a street vending plan based on which vending zones are earmarked (Step 9).
	\item Thirty-one percent of TVCs formed have complied with step 10 and published a street vendor charter.
\end{itemize}

\subsection*{26 States Have Drafted and Notified the Rules}
	The Act mandates, “The appropriate government shall, within one year from the date of commencement of this Act, by notification, make rules for carrying out the provisions of this Act.” The rules lay down the guidelines for implementation such as the minimum age of the vendor, processes for forming TVC, electing vendors representatives to the TVC, filing appeals in cases of disputes and for drafting and notifying the scheme.

	Four states—Arunachal Pradesh, Karnataka, Telangana and Nagaland—are yet to notify rules.

	Arunachal Pradesh repealed the Arunachal Pradesh Street Vendors (Protection of Livelihood and Regulation of Street Vending) Act, 2011, in March 2018 and came under the purview of the Central Act with effect from May 2018.\footnote{The Arunachal Pradesh Street Vendors (Protection of Livelihood and Regulation of Street Vending)(Repeal) Bill,2018} The repeal of the state Act and the adoption of Central Act partially explain the delay in notifying the rules. The state, however, already has TVCs in 45\% of its town. In the absence of the rules, the legal tenability of the committees is in question.

\subsection*{10 States Have Drafted and Notified the Scheme}
	Section 38 of the Act requires state governments to frame and notify a scheme within 6 months from May 2014, with due consultation from the local authority and the TVC. Eleven states—Arunachal Pradesh, Gujarat, Haryana, Karnataka, Kerala, Madhya Pradesh, Manipur, Nagaland, Puducherry, Sikkim and West Bengal—are yet to notify the scheme.

	The scheme is to provide the TVC with guidelines to conduct the survey of vendors and defines the manner and process for issuing identity cards and its content. TVCs have to follow the processes specified in the scheme to issue licences, mark vending zones, evict or relocate vendors and determine vending fees. A thoughtful and detailed scheme provides vendors with legal protection against informal governance practices—hafta collection, seizure of goods and sudden eviction.

\subsection*{Four States Have Formed Grievance Redressal Committees}
\addcontentsline{toc}{subsection}{Four States Have Formed Grievance Redressal Committees}
	Section 20 of the Act requires the formation of one or more Grievance Redressal Committees “consisting of a chairperson who has been a civil judge or a judicial magistrate and two other professionals.” Only four states—Assam, Madhya Pradesh, Uttarakhand and Punjab—have formed Grievance Redressal Committees.

\subsection*{Of the 7,263 Towns from 30 States, 33\% Have Formed Town Vending Committees}
\addcontentsline{toc}{subsection}{Of the 7,263 Towns from 30 States, 33\% Have Formed Town Vending Committees}
	Section 22(1) of the Act gives the power to the state government to decide the number of TVCs that may be  constituted under each local authority. We assume that ‘local authority’ refers to a ULB.

	15 states—Andhra Pradesh, Bihar, Chandigarh, Goa, Gujarat, Karnataka, Kerala, Meghalaya, Mizoram, Odisha, Puducherry, Punjab, Rajasthan, Telangana and Tripura—have formed TVCs in all their towns. Of these, Karnataka and Telangana do not have rules in places, and therefore the process with which the TVCs have been formed remains questionable.

	A TVC is responsible for conducting vendor enumeration and issuing identity cards and certificates of vending. This is the first step towards protecting and regulating vendors. Each year of delay in setting up TVCs imposes costs on vendors in the form of bribes to government officials, penalties during evictions, loss of livelihood in case of eviction and loss of goods during seizure. A study in 8 markets in Delhi estimated this loss at Rs 140 crore for 8,150 vendors annually \parencite{rattanpaper}.

\subsection*{Of the 2,382 Town Vending Committees Formed, 58\% Claim to Have Vendor Representation}
\addcontentsline{toc}{subsection}{Of the 2,382 Town Vending Committees Formed, 58\% Claim to Have Vendor Representation}
	The Act mandates at least 40\% vendor representation in any TVC to ensure participatory decision-making. Further, 1,391, accounting for 58\% of TVCs, have representation from vendors.

	Per the Act, the state is to define the manner and process of electing vendors in the rules. Some states, however, have nominated vendor representative. These include Himachal Pradesh\footnote{Section 5 of the Himachal Pradesh Street Vendors (Protection of Livelihood and Regulation of Vending) Act, 2014 Rules mandate, “the local authority/ municipality shall constitute the provisional Town Vending Committee, till such time, as the survey of street vendors is completed and election of the representatives of street vendors is held on the basis of such survey. The local authority shall nominate all the members of the provisional Town Vending Committee of various categories as required under the Act for this purpose.”} and Mizoram.\footnote{Mizoram has been following the Mizoram Street Vendor (Protection of Livelihood and Regulation of Vending) Act, 2011. The State Act was repealed in May, 2018. Section 4(2)(b) of the State Act mandates nomination of vendor representatives in the TVC such that they constitute 40\% of the committee. Based on this, vendors have been nominated in the former TVCs formed.}


	Nine—Andhra Pradesh, Chandigarh, Gujarat, Goa, Madhya Pradesh, Maharashtra, Odisha, Rajasthan and Uttar Pradesh—have elected representative.

\subsection*{Of the 2,382 Town Vending Committees Formed, 98\% Have Enumerated Vendors}
\addcontentsline{toc}{subsection}{Of the 2,382 Town Vending Committees Formed, 98\% Have Enumerated Vendors}
	Section 3 of the Act mandates the TVC to enumerate vendors in the manner and form prescribed in the state scheme. Also, 98\% of the 2,382 TVCs have completed enumeration of the vendors.

	Eight states—Arunachal Pradesh, Gujarat, Haryana, Karnataka, Kerala, Madhya Pradesh, Manipur, and Puducherry—however, have enumerated vendors without a scheme. In the absence of formal guidelines, the TVCs are not bound to issue public notices before the enumeration process. Poor information may result in vendor exclusion and force them to operate illegally.

\subsection*{Of the 2,382 Town Vending Committees Formed, 50\% Have Issued Identity Cards to 75\% of Identified Vendors}
\addcontentsline{toc}{subsection}{Of the 2,382 Town Vending Committees Formed, 50\% Have Issued Identity Cards to 75\% of Identified Vendors}

	A TVC, under Section 6 of the Act, is required to issue identity cards to every street vendor identified in the enumeration process.  Further, 50\% of all TVCs formed have issued identification to 75\% of enumerated vendors.


	In the absence clear guidelines for enumeration, governing bodies may apply numerical limits on licences encouraging extractive opportunities and limiting due process.

\subsection*{Twenty Percent of the TVCs Have A Published Street Vending Plan}
\addcontentsline{toc}{subsection}{Twenty Percent of the TVCs Have A Published Street Vending Plan}
	The local authority, under Section 21, is required to frame a plan of vending based on recommendations from the TVC. The plan covers elements such as criteria for earmarking no-vending zones, restricted zones, vending zones, and natural markets.

	Selected TVCs in Madhya Pradesh, Maharashtra, Meghalaya, Nagaland and Punjab have demarcated vending zones without a vending plan. The Act upholds certain principles for demarcating the vending zone. For example, a plan must "ensure that the provision of space or area for street vending is reasonable and consistent with existing natural markets." Without a vending plan, the basis of demarcation is not clear and there is no way to verify whether if it upholds the principles specified in the Act.

\subsection*{Thirty-One Percent of Town Vending Committees Have Published the Street Vendor Charter}
\addcontentsline{toc}{subsection}{Thirty-One Percent of Town Vending Committees Have Published the Street Vendor Charter}
	Section 26 of the Act mandates the TVC to publish a street vendor charter specifying the time of renewal of vendor identity cards and for maintaining updated records of registered street vendors.

	Further, 31\% of total TVCs formed have published the charter. These belong to seven states—Bihar, Madhya Pradesh, Maharashtra, Odisha, Puducherry, Rajasthan and Tamil Nadu.

\subsection*{Six States Have Assigned Office Space to Town Vending Committees}
\addcontentsline{toc}{subsection}{Six States Have Assigned Office Space to Town Vending Committees}
	According to Section 25 of the Act, “The local authority shall provide the Town Vending Committee with appropriate office space and such employees as may be prescribed.” Six states have assigned office space to TVCs (Table \ref{tab: OfficeSpace}).

%Table 2: States Where TVCs Have Assigned Office Space
\begin{table}[htpb]
\caption{States Where TVCs Have Assigned Office Space}
\label{tab: OfficeSpace}
\begin{tabular}{ l  r r r } %Alignment of Text
\toprule
States & \multicolumn{1}{p{9em}}{Assigned Office Space to TVCs} & \multicolumn{1}{p{9em}}{Number of TVCs with Assigned Office Space} & \multicolumn{1}{p{9em}}{Percentage of TVCs with Assigned Office Space}\\
\midrule
Chandigarh & 1 & 1 & 100\\
Madhya Pradesh & 364 & 58 & 16\\
Manipur & 28 & 6 & 21\\
Puducherry & 5 & 5 & 100\\
Punjab & 165 & 163 & 99\\
Rajasthan & 189 & 189 & 100\\
\end{tabular}
\end{table}

%===================WHICH STATES HAVE DONE BETTER===================
\section*{Which States Have Done Better Than The Others?}
\addcontentsline{toc}{section}{Which States Have Done Better Than The Others?}


	Despite the adoption of new ways to refine and strengthen the methodology, the index should be read with certain caveats. The data is self-reported by states. Moreover, in the absence of sufficient municipality information, it is, at best suggestive of the compliance and implementation at the local level. The Act creates scope for local participation, and a close look at the plans, schemes, rules, orders, circulars, and meeting minutes is necessary to comment on whether implementation is  promoting or suppressing justice. Finally, it does not capture state-specific idiosyncrasies that may influence state performance. For example, Mizoram, with the second highest score of 75, was one of the five states to have an Act on vending—the Mizoram Street Vendor (Protection of Livelihood and Regulation of Street Vending) Act, 2011—based on the model bill provided by the central government. The state Act was repealed in 2017. All orders and actions under the 2011 Act, however, are deemed to be issued under the provisions of the Central Act and, thus, are legally tenable. This gives Mizoram an advantage as some TVCs were already in place, and they remain valid even after the introduction of the Act.

	Despite the shortcomings, the index gives a bird’s eye view of implementation and offers a starting point for further enquiry. It induces federal competition between states and pushes them to outperform each other, especially if performance is tied to monetary grants.

%Table X: Compliance of States with the Act: Ranking Based on the Depth of Implementation of Each Step
\footnotesize
\begin{longtable}[l]{>{\raggedright}p{4cm}>{\raggedright\arraybackslash}p{10cm}}
\caption{Compliance of States with the Act: Ranking Based on the Depth of Implementation of Each Step}
\label{tab: Ranking}\\
	\toprule
	States & Insights \\
	\midrule
	\endfirsthead
	\toprule
	States & Insights \\
	\midrule
	\endhead
	\bottomrule
	\endfoot
	\endlastfoot
	\multicolumn{2}{c}{States with Best Compliance (Index Score Above 70)}\\
	\midrule
\cellcolor{SVACgreen1}\bf{Tamil Nadu}
\newline
Score: 76
\newline
Steps: 8/11
\cellcolor{SVACgreen2}The Act mandates the TVCs to enumerate vendors. The rules published by Tamil Nadu reiterated the mandates. "The survey of street vendors shall be carried out by the Town Vending Committee and completed within a period of six months from the date on which the scheme is notified."  482 TVCs have enumerated vendors in 664 town, out of the total 721 towns in the state.  It is possible that one TVC is administering enumeration of the vendors in more than one town.
\cellcolor{SVACgreen1}\bf{Mizoram}
\newline
\bf{Score: 75}
\newline
\bf{Steps: 8/11}
\cellcolor{SVACgreen2}All six towns have a TVC with vendor representatives. The TVCs, however, do not have an assigned space and have not published the charter. There are no grievance redressal committees.
\\
\cellcolor{SVACgreen1}\bf{Chandigarh}
\newline
\bf{Score: 75}
\newline
\bf{Steps: 8/11}
\\
\cellcolor{SVACgreen1}\bf{Rajasthan}
\newline
\bf{Score: 70}
\newline
\bf{Steps: 10/11}
&
\\
\midrule
\multicolumn{2}{l}{States with Good Compliance (Index score Between 50 to 70)}\\
\midrule
\cellcolor{SVACgreen3}\bf{Jharkhand}
\newline
\bf{Score: 69}
\newline
\bf{Steps: 8/11}
&
\\
\cellcolor{SVACgreen3}\bf{Himachal Pradesh}
\newline
\bf{Score: 69}
\newline
\bf{Steps: 8/11}
\cellcolor{SVACgreen2}Ninety-three percent of the towns have formed a TVC. Per the rules notified in December 2016, the TVCs are provisional and will be replaced once survey and election of vendors are complete. Until then, the vendor representatives are nominated.

Although TVCs are there only in 39 towns, vendor enumeration is complete in 41 towns. It is possible that one TVC is administering enumeration in more than one town.
\\
\cellcolor{SVACgreen3}\bf{Uttar Pradesh}
\newline
\bf{Score: 67}
\newline
\bf{Steps: 8/11}
\cellcolor{SVACgreen2}Thirty out of 130 towns have formed a TVC with vendor representation. The Uttar Pradesh Street Vendors Rules (2017) call for registered associations of street vendors to apply for membership in a TVC. The members of the vendor association elect the representative from the association. If the number of applicants is higher than required, the municipality conducts a lottery. The 30 TVCs have enumerated vendors and issued identity cards. Also, 13 TVCs have earmarked vending zones in the absence of a plan.
\\
\cellcolor{SVACgreen3}\bf{Puducherry}
\newline
\bf{Score: 66}
\newline
\bf{Steps: 8/11}
&
\cellcolor{SVACgreen2}Puducherry has complied fully with all steps except three: notification of scheme, constitution of a Grievance Redressal Committee, and appointment of vendor representatives in the TVCs. Enumeration, registration and demarcation of vending zones without vendor representatives in the TVC defeat the purpose a participatory committee.
\\
\cellcolor{SVACgreen3}\bf{Punjab}
\newline
\bf{Score: 65}
\newline
\bf{Steps: 9/11}
&
\cellcolor{SVACgreen2}Punjab is one of the four states that have formed the Grievance Redressal Committee.

The Act mandates the state to define the process of election for representing vendors in the TVC. Punjab has used “show of hands” as the way to elect vendors. On the basis of  this, all 165 towns have vendor representation in the TVCs.


In the absence of a vending plan, vending zones have been earmarked in three towns.
\\
\cellcolor{SVACgreen3}\bf{Odisha}
\newline
\bf{Score: 60}
\newline
\bf{Steps: 9/11}
&
\cellcolor{SVACgreen2}Odisha has implemented 9 out of 11 steps: all 105 towns have formed a TVC with elected vendor representation; 79\% of the TVCs have enumerated vendors; 14 out of the TVCs have issued identity cards and 5 percent of the towns have published a plan and earmarked vending zones.
\\
\cellcolor{SVACgreen3}\bf{Goa}
\newline
\bf{Score: 60}
\newline
\bf{Steps: 8/11}
&
\\
\cellcolor{SVACgreen3}\bf{Andhra Pradesh}
\newline
\bf{Score: 59}
\newline
\bf{Steps: 8/11}
&
\cellcolor{SVACgreen2}In Andhra Pradesh 1 town has published the plan but 40 TVCs have earmarked vending zones. It is not clear whether one plan is being used to earmark vending zones in all 40. If not, then how have TVCs earmarked no-vending, restricted vending and vending zones? What are the spatial planning norms used?
\\
\cellcolor{SVACgreen3}\bf{Gujarat}
\newline
\bf{Score: 53}
\newline
\bf{Steps: 7/11}
&
\cellcolor{SVACgreen2}Gujarat has notified the rules not the scheme. A scheme is to specify the manner of conducting the survey and the form and manner of issuing ID cards. In the absence of a scheme, 168 out of 169 TVCs have already completed the enumeration exercise and issued identification to more than 75 percent vendors.
\\
\cellcolor{SVACgreen3}\bf{Telangana}
\newline
\bf{Score: 53}
\newline
\bf{Steps: 5/11}
&
\cellcolor{SVACgreen2}Due to the dissolution of the state government in early 2017, there has been a delay in notifying the rules. However, the state has published the scheme.

Seventy-four towns in the state shave formed 103 TVCs with elected vendors.
\\
\midrule
\multicolumn{2}{l}{States with Moderate Compliance (Index Score Between 50 to 59)}\\
\midrule
\cellcolor{SVACyellow1}\bf{Bihar}
\newline
\bf{Score: 47}
\newline
\bf{Steps: 8/11}
&
\cellcolor{SVACyellow2}All 144 towns in Bihar have formed a TVC without vendor representation. Only 3 towns have published the plan but 46 TVCs have already earmarked vending zones.
\\
\cellcolor{SVACyellow1}\bf{Uttarakhand}
\newline
\bf{Score: 47}
\newline
\bf{Steps: 6/11}
&
\cellcolor{SVACyellow2}In Uttarakhand, 40 out of 93 towns have a TVC. Vendor enumeration is complete in 55 towns. 10 TVCs have issued identity cards to more than 75 percent of vendors. However, none of the TVCs have vendor representatives so far.
\\
\cellcolor{SVACyellow1}\bf{Madhya Pradesh}
\newline
\bf{Score: 46}
\newline
\bf{Steps: 8/11}
&
\cellcolor{SVACyellow2}Madhya Pradesh has implemented 8 out of 11 steps in less than a third of the towns.
\\
\cellcolor{SVACyellow1}\bf{Haryana}
\newline
\bf{Score: 46}
\newline
\bf{Steps: 5/11}
&
\cellcolor{SVACyellow2}In Haryana, there are 76 TVCs for 80 towns. None of the TVCs have vendor representatives. The state has not notified the scheme but has completed enumeration in all towns; 58 towns have a plan but 72 have earmarked vending zones.
\\
\cellcolor{SVACyellow1}\bf{Chhattisgarh}
\newline
\bf{Score: 36}
\newline
\bf{Steps: 5/11}
&
\cellcolor{SVACyellow2}Sixty four out of 169 towns in Chhattisgarh have a TVC but without vendor representation. All 64 TVCs have completed the survey. Also, 35 TVCs have issued ID cards to more than 75 percent of identified vendors.
\\
\cellcolor{SVACyellow1}\bf{Meghalaya}
\newline
\bf{Score: 43}
\newline
\bf{Steps: 6/11}
&
\cellcolor{SVACyellow2}In Meghalaya, there are seven TVCs but without vendor representation. One town has marked vending zones but without a plan.
\\
\cellcolor{SVACyellow1}\bf{Tripura}
\newline
\bf{Score: 40}
\newline
\bf{Steps: 4/11}
&
\cellcolor{SVACyellow2}Tripura has formed TVCs in all 20 towns but only 25 percent of the TVCs have enumerated vendors.
\\
\cellcolor{SVACyellow1}\bf{Assam}
\newline
\bf{Score: 39}
\newline
\bf{Steps: 5/11}
&
\cellcolor{SVACyellow2}Assam is one of the four states that have formed Grievance Redressal Committees.
\\
\cellcolor{SVACyellow1}\bf{Kerala}
\newline
\bf{Score: 37}
\newline
\bf{Steps: 4/11}
&
\cellcolor{SVACyellow2}Kerala has a TVC in all its towns but without vendor representation. Vendor enumeration is complete in all TVCs and 35 percent of these TVCs have issued identity cards, without a scheme in place.
\\
\cellcolor{SVACyellow1}\bf{Arunachal Pradesh}
\newline
\bf{Score: 34}
\newline
\bf{Steps: 7/11}
&
\cellcolor{SVACyellow2}Arunachal Pradesh is one of the four states without rules in place. The delay in the implementation may be attributed to the process to repeal the Arunachal Pradesh Street Vendors Act, 2011. However, the repeal Act saves all orders and action taken under the 2011 Act indicating that the state had not progressed significantly under the previous Act.
\\
\cellcolor{SVACyellow1}\bf{Maharashtra}
\newline
\bf{Score: 32}
\newline
\bf{Steps: 7/11}
&
\cellcolor{SVACyellow2}Ninety-seven out of 3799 towns have formed a TVC.
\\
\midrule
\multicolumn{2}{l}{States with Poor Compliance (Index Score Between 30 to 49)}\\
\midrule
\cellcolor{SVACred1}\bf{Manipur}
\newline
\bf{Score: 29}
\newline
\bf{Steps: 7/11}
&
\cellcolor{SVACred2}Manipur has enumerated vendors, issued identity cards, published a plan and earmarked vending zones but in the absence of a scheme.
\\
\cellcolor{SVACred1}\bf{Karnataka}
\newline
\bf{Score: 23}
\newline
\bf{Steps: 4/11}
&
\cellcolor{SVACred2}Karnataka has formed 265 TVCs in 237 states. The TVCs have enumerated vendors as well. However, the state has not yet notified the rules or the scheme, raising questions on the tenability of the progress.
\\
\cellcolor{SVACred1}\bf{Sikkim}
\newline
\bf{Score: 21}
\newline
\bf{Steps: 2/11}
&
\cellcolor{SVACred2}Sikkim has not yet published the scheme.
\\
\cellcolor{SVACred1}\bf{West Bengal}
\newline
\bf{Score: 13}
\newline
\bf{Steps: 2/11}
&
\cellcolor{SVACred2}West Bengal has notified the rules and formed TVCs in 3 out of 239 towns, accounting for 1\%, the lowest among all states.
\\
\midrule
\multicolumn{2}{l}{States with Very Poor Compliance (Index Score Less Than 29)}\\
\midrule
\cellcolor{SVACred3}\bf{Nagaland}
\newline
\bf{Score: 3}
\newline
\bf{Steps: 2/11}
&
\cellcolor{SVACred2}Nagaland has not notified rules and scheme. It has formed TVCs in 2 out of 11 towns and has earmarked vending zones in the two areas.\\
\bottomrule
	\end{longtable}
            \scriptsize
             \begin{landscape}
  \rowcolors{2}{gray!25}{white}
  \centering
            \begin{longtable}{p{2cm}p{0.5cm}p{0.5cm}>{\raggedleft}p{0.5cm}>{\raggedleft}p{0.5cm}>{\raggedleft}p{0.5cm}>{\raggedleft}p{0.5cm}>{\raggedleft}p{0.5cm}>{\raggedleft}p{0.5cm}>{\raggedleft}p{0.5cm}>{\raggedleft}p{0.5cm}>{\raggedleft}p{1.0cm}>{\raggedleft}p{0.5cm}>{\raggedleft}p{0.5cm}>{\raggedleft}p{0.5cm}>{\raggedleft}p{0.5cm}>{\raggedleft}p{0.5cm}>{\raggedleft}p{0.5cm}>{\raggedleft}p{0.5cm}>{\raggedleft}p{0.5cm}>{\raggedleft}p{0.5cm}>{\raggedleft\arraybackslash}p{0.5cm}}
            \caption{Step-Wise Data on Implementation for 30 states, as of October 31 2019}
            \label{tab: Index}\\
States &
\rotatebox{90}{Rules} &
\rotatebox{90}{Scheme} &
\rotatebox{90}{Towns} &
\rotatebox{90}{Total TVCs} &
\rotatebox{90}{\thead{Total TVCs/ \\ Total towns (\%)}} &
\rotatebox{90}{\thead{Have vendor \\ representatives}} &
\rotatebox{90}{\thead{\% of TVCs with \\ vendor representatives}} &
\rotatebox{90}{Completed enumeration} &
 \rotatebox{90}{\thead{\% of TVCs that \\ completed enumeration}} &
 \rotatebox{90}{\thead{Issued IDs to \textgreater 75\% \\ vendors}} &
 \rotatebox{90}{\thead{\% of TVCs that \\ issued ID \textgreater  75\% of \\ identified vendors}} &
 \rotatebox{90}{Published plan} &
 \rotatebox{90}{\thead{\% of TVCs with \\ published plan}} &
 \rotatebox{90}{Vending zones} &
 \rotatebox{90}{\thead{\% of TVCs that marked \\ vending zones}} &
 \rotatebox{90}{Published charter} &
 \rotatebox{90}{\thead{\% of TVCs that \\ published charter}} &
 \rotatebox{90}{Assigned office space} &
 \rotatebox{90}{\thead{\% of TVCs that have \\ assigned office space}} &
 \rotatebox{90}{\# of GRCs in the state} &
 \rotatebox{90}{\% of towns with GRC} \\

AP & Y & Y & 110 & 110 & 100 & 110 & 100 & 110 & 100 & 85 & 77 & 1 & 1 & 50 & 45 & 0 & 0 & 0 & 0 & 0 & 0 \\
AR & N & N & 33 & 15 & 45 & 10 & 67 & 14 & 93 & 14 & 93 & 8 & 53 & 4 & 27 & 0 & 0 & 0 & 0 & 0 & 0 \\
AS & Y & Y & 110 & 24 & 22 & 0 & 0 & 24 & 100 & 0 & 0 & 0 & 0 & 0 & 0 & 0 & 0 & 0 & 0 & 4 & 4 \\
BR & Y & Y & 144 & 144 & 100 & 0 & 0 & 46 & 32 & 42 & 29 & 3 & 2 & 46 & 32 & 42 & 29 & 0 & 0 & 0 & 0 \\
CH & Y & Y & 1 & 1 & 100 & 1 & 100 & 1 & 100 & 0 & 0 & 1 & 100 & 1 & 100 & 0 & 0 & 1 & 100 & 0 & 0 \\
CG & Y & Y & 169 & 64 & 38 & 0 & 0 & 64 & 100 & 35 & 55 & 0 & 0 & 0 & 0 & 0 & 0 & 0 & 0 & 0 & 0 \\
GA & Y & Y & 14 & 14 & 100 & 14 & 100 & 11 & 79 & 2 & 14 & 4 & 29 & 4 & 29 & 0 & 0 & 0 & 0 & 0 & 0 \\
GJ & Y & N & 169 & 169 & 100 & 12 & 7 & 168 & 99 & 168 & 99 & 131 & 78 & 131 & 78 & 0 & 0 & 0 & 0 & 0 & 0 \\
HR & Y & N & 80 & 80 & 100 & 0 & 0 & 80 & 100 & 0 & 0 & 58 & 73 & 72 & 90 & 0 & 0 & 0 & 0 & 0 & 0 \\
HP & Y & Y & 54 & 39 & 72 & 39 & 100 & 40 & 103 & 9 & 23 & 31 & 79 & 25 & 64 & 0 & 0 & 0 & 0 & 0 & 0 \\
JH & Y & Y & 48 & 44 & 92 & 44 & 100 & 44 & 100 & 22 & 50 & 21 & 48 & 34 & 77 & 0 & 0 & 0 & 0 & 0 & 0 \\
KA & N & N & 271 & 265 & 98 & 0 & 0 & 265 & 100 & 0 & 0 & 4 & 2 & 4 & 2 & 0 & 0 & 0 & 0 & 0 & 0 \\
KL & Y & N & 93 & 93 & 100 & 0 & 0 & 93 & 100 & 33 & 35 & 0 & 0 & 0 & 0 & 0 & 0 & 0 & 0 & 0 & 0 \\
MP & Y & N & 364 & 58 & 16 & 58 & 100 & 58 & 100 & 0 & 0 & 0 & 0 & 58 & 100 & 4 & 7 & 58 & 100 & 3 & 1 \\
MH & Y & Y & 3799 & 97 & 3 & 24 & 25 & 17 & 18 & 0 & 0 & 0 & 0 & 8 & 8 & 7 & 7 & 0 & 0 & 0 & 0 \\
MN & Y & N & 28 & 6 & 21 & 0 & 0 & 2 & 33 & 1 & 17 & 2 & 33 & 2 & 33 & 0 & 0 & 6 & 100 & 0 & 0 \\
ML & Y & Y & 7 & 7 & 100 & 0 & 0 & 3 & 43 & 1 & 14 & 0 & 0 & 1 & 14 & 0 & 0 & 0 & 0 & 0 & 0 \\
MZ & Y & Y & 6 & 6 & 100 & 6 & 100 & 6 & 100 & 5 & 83 & 5 & 83 & 5 & 83 & 0 & 0 & 0 & 0 & 0 & 0 \\
NL & N & N & 11 & 2 & 18 & 0 & 0 & 0 & 0 & 0 & 0 & 0 & 0 & 2 & 100 & 0 & 0 & 0 & 0 & 0 & 0 \\
OD & Y & Y & 112 & 104 & 93 & 104 & 100 & 111 & 107 & 21 & 20 & 5 & 5 & 5 & 5 & 15 & 14 & 0 & 0 & 0 & 0 \\
PY & Y & N & 5 & 5 & 100 & 0 & 0 & 5 & 100 & 5 & 100 & 5 & 100 & 5 & 100 & 5 & 100 & 5 & 100 & 0 & 0 \\
PB & Y & Y & 165 & 165 & 100 & 165 & 100 & 163 & 99 & 61 & 37 & 0 & 0 & 3 & 2 & 0 & 0 & 163 & 99 & 13 & 8 \\
RJ & Y & Y & 189 & 189 & 100 & 189 & 100 & 189 & 100 & 85 & 45 & 7 & 4 & 126 & 67 & 1 & 1 & 189 & 100 & 0 & 0 \\
SK & Y & N & 4 & 3 & 75 & 0 & 0 & 0 & 0 & 0 & 0 & 0 & 0 & 0 & 0 & 0 & 0 & 0 & 0 & 0 & 0 \\
TN & Y & Y & 721 & 482 & 67 & 482 & 100 & 664 & 138 & 482 & 100 & 164 & 34 & 0 & 0 & 664 & 138 & 0 & 0 & 0 & 0 \\
TS & N & Y & 74 & 103 & 139 & 103 & 100 & 66 & 64 & 72 & 70 & 0 & 0 & 0 & 0 & 0 & 0 & 0 & 0 & 0 & 0 \\
TR & Y & Y & 20 & 20 & 100 & 0 & 0 & 5 & 25 & 0 & 0 & 0 & 0 & 0 & 0 & 0 & 0 & 0 & 0 & 0 & 0 \\
UP & Y & Y & 130 & 30 & 23 & 30 & 100 & 30 & 100 & 30 & 100 & 16 & 53 & 29 & 97 & 0 & 0 & 0 & 0 & 0 & 0 \\
UK & Y & Y & 93 & 40 & 43 & 0 & 0 & 55 & 138 & 10 & 25 & 0 & 0 & 0 & 0 & 0 & 0 & 0 & 0 & 1 & 1 \\
WB & Y & N & 239 & 3 & 1 & 0 & 0 & 0 & 0 & 0 & 0 & 0 & 0 & 0 & 0 & 0 & 0 & 0 & 0 & 0 & 0 \\
\midrule
Sum Total / \% & 0 & 0 & 7263 & 2382 & 33 & 1391 & 58 & 2334 & 98 & 1183 & 50 & 466 & 20 & 615 & 26 & 738 & 31 & 422 & 18 & 21 & 13 \\
No. of States that Initiated Work & 24 & 0 &  & 30 &  & 16 &  & 27 &  & 20 &  & 17 &  & 21 &  & 7 &  & 6 &  & 4 & 4 \\

\end{longtable}
\end{landscape}
%===================LESSONS FROM THE FIELD==========================
\section*{Closer to the Field: Challenges in Implementation}
\addcontentsline{toc}{section}{Closer to the Field: Challenges in Implementation}

\begin{quote}
\end{quote}





The objective is to describe the constitution of the TVC, the relations between its unequal actors and how it functions. The case studies do not offer explicit answers but raise pertinent questions on how the citizen-driven democratic governance shapes the city.

\subsection*{Gurugram: Hired Private Agencies to Manage and Regulate Vendors}
\addcontentsline{toc}{subsection}{Gurugram: Hired Private Agencies to Manage and Regulate Vendors}

Gurugram, home to second largest number of street vendors among 80 towns in Haryana, has identified 18,670 vendors.\footnote{As stated by the State Urban Development Authority, Haryana (SUDA-H) under ‘Support to Urban Street Vendors (SUSV) in RTI response dated 21 November 2018} The Millennium City, with a population of more than 15 lakhs, has grown from housing 800 start-ups in 2015 to more than 1500 in 2018 \parencite{vermanews}. With the visual backdrop of modern residential buildings and corporate houses coexisting with village settlement areas, we study the role of vendors and their rights to the city.
Between November and December 2018, we conducted interviews with seven members of the TVC including representatives from an NGO (one), a private enterprise (one), vendor associations (two), and government officials (three). Eight members declined the request for interview. The list of interviewees includes the following:

\begin{itemize}
\item \textbf{TVC members:}
	\begin{itemize}
	\item \textbf{Private enterprise contracted:} Spick and Span Services Pvt Ltd. (SSSPL)
	\item \textbf{Representatives from two vendor associations:} Dron Rehri Patri Welfare Association and Jan Kalyan Sangh
	\item \textbf{NGO:} Bringing Smiles
	\end{itemize}
\item \textbf{Non-TVC:} Fifteen vendors from areas with high vendor population density such as Sadar Bazaar, Sector 5, Sector 14, HUDA City Centre and MG Road Metro. Sectors 5 and 14 are earmarked vending zones.
\end{itemize}

The study describes three aspects of implementation: vendor representation in TVC; contractual arrangement with private enterprises to perform TVC functions; and the lack of a redressal mechanism for vendors.
\begin{longtable}[l]{>{\raggedright}p{7cm}>{\raggedright\arraybackslash}p{7cm}}
  \caption{Implementation of the Street Vendors Act 2014 in Gurugram - at a Glance}\\
    \toprule
    Step & Status\\
    \midrule
    \endfirsthead
    \toprule
    Step & Status\\
    \midrule
    \endhead
    \bottomrule
    \endfoot
    \endlastfoot
    \multicolumn{2}{p{36.055em}}{Haryana} \\
    \midrule
    \rowcolor{SVACgreen2}State government to draft and notify rules & Complete, notified on 31 Jan 2017 \\
    \rowcolor{SVACyellow2} State government to draft and notify the scheme & Drafted, in the process for approval \\
    \rowcolor{SVACred2} State government to form the Grievance Redressal Committee & No \\
    \midrule
    \multicolumn{2}{p{36.055em}}{Gurugram}\\
    \midrule
    \rowcolor{SVACgreen2} Local authority to form the TVC & The city has constituted one TVC in its one municipality (MCG) \\
    \rowcolor{SVACyellow2} TVC to conduct a survey of street vendors & Ongoing (more than 18,000+) \\
    \rowcolor{SVACyellow2} TVC to issue identity cards to street vendors & Ongoing (14000+) \\
    \rowcolor{SVACyellow2}TVC to earmark vending zones & Ongoing (49 zones earmarked) \\
    \rowcolor{SVACred2} Local authority to draft and publish a street vending plan & No response \\
    \bottomrule
\end{longtable}

\subsection*{Mandate of the Act v/s Mandate of the MCG: Only 14\% Vendor Representation in TVC, Instead of 40\%}
\addcontentsline{toc}{subsection}{Mandate of the Act v/s Mandate of the MCG: Only 14\% Vendor Representation in TVC, Instead of 40\%}

\begin{wrapfigure}{r}{0.6\textwidth}
\centering
\includegraphics[width=0.6\textwidth]{mandatevsreality.pdf}

The Act requires each TVC, headed by the Municipal Commissioner, to have at least 10\% representation from NGOs or community-based organisations and 40\% from vendors. Other members may include representation from the local authority, traffic or local police, banks, vendors, market associations and resident welfare associations (RWAs). Per the order from MCG dated 7 August 2018, vendors account for only 14\% of the total TVC strength of 29 members (Appendix \ref{sec: Appendix 6}).


\subsection*{Mandate v/s Practice: Only 3 out of 17 Attendees Represented Vendors at the TVC Meeting, Resulting in Biased Decisions}
\addcontentsline{toc}{subsection}{Mandate v/s Practice: Only 3 out of 17 Attendees Represented Vendors at the TVC Meeting, Resulting in Biased Decisions}


Separately, we also reached out to 15 of the 23 signatories to verify their presence at the meeting. Three denied being a part of any such meeting. One NGO member, on the condition of anonymity, mentioned that he was only present as he is “friends with the City Project Officer.”{\footnote{Respondent number 15, 21 and 22 from the list in Appendix 7 denied being a part of the TVC; one, being a representative from the market welfare association and second, the ad-hoc pradhan of Sector 14.}

\subsubsection*{Who Wins the Argument: Disagreement Between Government Officials and Vendors on Vending Sites}


The meeting on 21 August 2018 was called to “take fair decision… regarding shifting of vending zones after hearing arguments of all concerned parties/stakeholders.” This was after three complaints by vendors to different authorities for reconsideration of the decision to withdraw sites. Out of the 17 members who voted on the decision, only 3 opposed. All three members were representatives of vendors. The minutes conclude by referring to the “democratically expressed views of the majority” and issued an order to shift vendors with “immediate effect.”

This example of the withdrawal of two sites from Sector 14 reflects how vendors may be overpowered easily in the absence of sufficient representation and a genuine will to formalise vendors. The effort here, it seems, is to obtain the majority to agree to a decision already decided. The meeting minutes repeatedly referred to the problems caused by vendors but were silent on the argument raised by vendors.



\begin{wrapfigure}{r}{0.65\textwidth}
\centering
\includegraphics[height=1.15\textwidth]{WhoDecides.pdf}

Although the Act opens channels and creates a super structure for negotiation between government and civil society, the devil lies in the detail. The suppression or protection of vendors will depend on the quality of regulations proposed by the local authority. The Act requires the plan for street vending to recognise the existing markets where buyers and sellers congregate. For any spatial planning exercise to be successful, it must take into consideration the commercial viability of vending sites. A thoughtful consideration of the existing patterns of vending and buy-in from vendors is necessary to ensure that the demarcation of zones does not become a futile exercise.

\subsection*{Contracting out TVC functions: Enabling or disabling Governance?}
\addcontentsline{toc}{subsection}{Contracting out TVC functions: Enabling or disabling Governance?}

\begin{itemize}
\item Spatial planning taking into account natural markets, weekend markets, weekly haats, and night bazaars;
\item Demarcation of vending sites;
\item Design of carts to optimise space, keeping “aesthetics” into consideration;
\item Exhibit regulation and management of vendors;
\item Proposal for solid waste management;
\item Enumeration in allocated zones and;
\item Monitoring food adulteration and compliance with Food Safety and Standards Authority of India (FSSAI) norms.
\end{itemize}


\subsubsection*{Multiple Surveys, Multiple Identities?}
\addcontentsline{toc}{subsection}{Multiple Surveys, Multiple Identities?}

In Gurugram, the enumeration of vendors was first done in 2014. The exercise identified more than 14,000 vendors in all 35 wards. In 2016, the three private firms shortlisted for the regulation and management of vendors re-surveyed the sites to ensure that all vendors in the sectors allocated to them were enlisted. In 2018, the Haryana government started a new exercise that aspired to cover vendors of the whole state. \\

\begin{mdframed}[backgroundcolor=gray!20]

While she was allowed to reclaim her goods, she still can not vend for the fear of being evicted or harassed. In the absence of a one-stop grievance redressal committee, many vendors like her struggle to find the right platform to voice their concerns.
\end{mdframed}



\subsubsection*{Fees and Fines: What Does the Vendor Earn and Pay?}

\begin{wrapfigure}{l}{0.65\textwidth}
\centering
\includegraphics[width=0.65\textwidth]{"ggncart".png}

The scheme for street vendors must provide for “the manner of collecting ...vending fees, maintenance charges and penalties for registration.”

Vendors in Gurugram, under the new policy, pay a one-time cart fee of Rs 85,000 to 1.5 lakh. The fee is decided based on the income of the vendor, and the nature of vending. It also varies depending on the the private enterprise in charge. The TVC meeting held in June 2016 discussed the possible ways of vendors to pay for the cart. The local authority officials refused to grant advertising rights to private enterprises; they were willing to provide carts free of cost to vendors, if such rights were accorded. Private parties alternatively offered to issue loans through banks/microcredit institutions to vendors in case they could not self-finance. Vendor representatives agreed stating that “street vendors are being harassed at multiple levels and they are willing to invest in anything which renders them recognition and security of tenure at a given location with the intent to safeguard their interest.”



The costs imposed on vendors and its impact are pertinent subjects but outside the scope of this study. Some of the questions for further research are as follows: Are vendors able to afford the “aesthetic” carts? How many vendors have been issued loans? What if vendors fail to replay? How are the funds generated through the collection of maintenance and other fees used?

\subsection*{No Mechanism for Dispute Redressal, Despite Piling Complaints}
\addcontentsline{toc}{subsection}{No Mechanism for Dispute Redressal, Despite Piling Complaints}





%===================CONCLUSIONS/RECOMMENDATIONS==================
\newpage
\section*{Conclusion}
\addcontentsline{toc}{section}{Conclusion}



There are three parts to the report: a look at the interpretation of the Act by the Higher Courts, a statistical capture of the progress by states in implementing the Act, and a case study of two urban cities to explore how the new Act is reshaping urban space management.

\subsection*{Role of Courts: No Protection in Times of Tardy Implementation}


\subsection*{Progress by State Governments: Slow to Show}
Monitoring state progress on compliance helps ensure we have the systems in place necessary to bring reform. In 2018, we extended our work from 2017 on measuring absolute and relative progress on state compliance based on self-reported data. Out of the 30 states analysed, we find that 11 are yet to notify the scheme—the statutory deadline for which was October 2014. Of the 7,263 towns from 30 states, over 65\% are yet to form a TVC. Only 58\% of these TVCs have vendor representatives. Any functions performed by a TVC without vendor representatives may be legally untenable since they violate the mandate of the Act.

To measure relative compliance, we give each state a score based on their performance on 11 steps encompassing notification of rules and schemes, formation of TVCs with vendor representation, enumeration of vendors, issuance of vending certificates, assignment of office space, demarcation of vending zones, and publication of charter and vending plans. We have assigned a higher weightage to rule-making and creation of institutions—without which implementation can be and has been questioned in the Court. Tamil Nadu, Mizoram, Chandigarh and Rajasthan have performed the best. Nagaland, West Bengal, Sikkim, Karnataka and Manipur lag far behind. West Bengal has 3 TVCs for 239 towns and Nagaland has 2 TVCs for 11 towns. Both states have only implemented two out of 11 steps of implemented.

\subsection*{Up Close: New Structures Attempt to Suppress}


Gurugram is unique in its approach to contract selected functions of the TVC to private firms, raising questions on state capacity to use in-house resourced to enumerate and certify vendors.
In Delhi, … (yet to add)

In SumThe Act has the potential to redefine historically ambivalent vendors’ rights to  city’s public spaces. Success, however, rests on the capacity and willingness of vendors and state administrators to cooperate. In absence of penalties for non-compliance, progress at the state level has been sluggish. As the rules and schemes are being notified, the next step is to evaluate the details: whether the new regulatory tools are used to prioritise or delegitimize vendors’ rights. Unless we establish rule of law for the use of public space, mechanisms to redress grievances and build the capacity of vendors to negotiate with administrators as they balance the many citizens rights to livelihood, safety, essential urban services, and public commons—the street will continue to seed conflict.


%===================BIBLIOGRAPHY====================================
\newpage
\section*{Bibliography}
\addcontentsline{toc}{section}{Bibliography}

%===================APPENDICES======================================
%Appendix 1
\section*{Appendix 1: References for Legal Analysis}
\addcontentsline{toc}{section}{Appendix 1: References for Legal Analysis}
\label{sec: Appendix 1}
            \scriptsize
              \rowcolors{3}{gray!25}{white}
            \begin{longtable}{>{\raggedright}p{1.5cm}>{\raggedright}p{2.5cm}>{\raggedright}p{1.3cm}>{\raggedright}p{1.5cm}>{\raggedright}p{1.1cm}>{\raggedright}p{1.2cm}>{\raggedright}p{1cm}>{\raggedright}p{1.8cm}>{\raggedright}p{1.3cm}>{\raggedright}p{4.45cm}>{\raggedright\arraybackslash}p{1.2cm}}

            \caption{Street Vendor Case Data}\\

Case Citation &
Case Title &
Case No. &
Court &
State &
Date of \\ Judgement &
Relevant \\ (Y/N) &
Issue &
Outcome &
Decisions/Rule Laid Down &
Significance (1-5) \footnotemark \\
\midrule
\endfirsthead
%\rotatebox[origin=c]{90}
Case Citation &
Case Title &
Case No. &
Court &
State &
Date of \\ Judgement &
Relevant \\ (Y/N) &
Issue &
Outcome &
Decisions/Rule Laid Down &
Significance (1-5) \\
\midrule
\endhead
\bottomrule
\endfoot
\bottomrule
\endlastfoot
\footnotetext{Significance indicated the likelihood of a case to be quoted and cited in subsequent decisions.}

MANU/TN/10\\57/2018 & A.T. Mhammed Ismail and Ors v. The Commissioner, Greater Chennai Corporation and Ors & WP Nos. 4171 and 4172 of 2018 & High Court of Madras & Tamil Nadu & 2/26/2018 & Y & Eviction & Deferred  & When statute contemplates the issuance of certificate and right to carry on the business of street vending in accordance with the terms and conditions mentioned in the certificate of vending, and when a representation is made by street vendors, it is the duty of the competent authority to consider such representation and pass appropriate orders. & 2  \\

MANU/KE/20\\75/2018 & Abbas V. and Ors v. State of Kerala and Ors  & WP(C) No. 25239 of 2018 & High Court of Kerala & Kerala & 7/26/2018  & Y & Eviction; due process; relocation; legislative overlap  & Deferred & Competent authority to consider applications after hearing the petitioners without much delay, preferably within 2 months. (Approach the highway authority u/s 4 of the Kerala Highway Protection Act to seek permission for doing business. & 2 \\

MANU/TN/54\\75/2018 & D.S.Sundar v. The Special Commissioner for Handicapped and Ors & WPC 16054 of 2013 & High Court of Madras & Tamil Nadu & 9/20/2018 & Y & Eviction; definition of street vendor (whether a bunk owner is a street vendor) & Deferred & The petitioner not to be uprooted unless he parks his \textit{Rehri} in the parking area or hinders traffic. & 2 \\

2017 SCC ONLINE P\&H4106 & Deepu Sharma v. State of Punjab & CWP 5816 of 2015 (O\&M) & High Court of Punjab \& Haryana & Punjab & 9/7/2017 & Y & Harassment; eviction; survey & Directions & (1) Survey already conducted. (2) The petitioner to approach the TVC to consider all other aspects including public utility and other practical difficulties. & 2 \\

MANU/DE/23\\11/2018 & \textit{Delhi Pradesh Rehri Patri Khomcha Hawkers Union and Ors} v. South Delhi Municipal Corporation and Ors & WP(C) 6672-6673/2018 & High Court of Delhi & Delhi &7/3/2017  & Y & Elections; display the zone-wise voter list, vote-applicant list, etc. & Partially dismissed & - \quotes{...In case the petitioners were aggrieved by the procedure which was being followed by
the Corporations, they should have approached this Court at the earliest point of
time. \[...\] - We are also of the view that in case objections are invited, at this stage, it
would not only result in elections being postponed but would also be in violation of
the orders passed by the Supreme Court of India.} & 3 \\

2018 SCC ONLINE DEL 8251 & G. Raju v. North Delhi Municipal Corporation & WP(C) 2588 of 2018 & High Court of Delhi &  Delhi & 3/19/2018 & Y & Eviction & Deferred & Directions:

(i)The petitioner to approach TVC as and when constituted with all the supporting documents.

(ii) TVC to consider the documents in accordance with the law expeditiously.

(iii) Merely because the petitioner is not found vending at the site when the survey is conducted, that by itself would not be a ground alone to reject his case. & 1 \\

MANU/KE/21\\26/2018 & K.H. Abdul Rahman v. District Industries Centre and Ors & WP(C) 3108/2018 & High Court of Kerala &  Kerala & 8/10/2018 & Y & Eviction; CoV & Deferred  & Deferred to TVC & 1 \\

2017 SCC ONLINE KER 9727 & Karutha Lakshmi v. District Collector, Civil Station, Kannur & WPC 28290/2016 & High Court of Kerala & Kerala & 3/2/2017 & Y & Due process; eviction & Deferred  & An opportunity of being heard to be provided to the petitioner & 2 \\

2017 SCC ONLINE MAD 31753 & Marimuthu v. Corporation of Chennai & WP 26333/2017 & High Court of Madras & Tamil Nadu & 10/12/2017 & Y & Eviction; CoV & Deferred  & Representation/Request to be considered by TVC within a period of 1 month &  1 \\

2018 SCC ONLINE DEL 6476 & Mukesh Kumar v. New Delhi Municipal Council and Anr & WP(C) 61/2018 & High Court of Delhi & Delhi & 1/5/2018 & Y & Eviction & Deferred  & (1) The petitioner would approach the TVC as and when it is constituted with all the supporting documents.

(2) TVC will consider the case of the petitioner in accordance with law and expeditiously after considering all the material on record.

(3) Merely because the petitioner is not found vending at the site when the survey is conducted, that by itself would not be a ground alone to reject his case. & 1 \\

2017 SCC ONLINE KER 34700 & Murali v. State of Kerala & WP(C) 32001/2017 & High Court of Kerala & Kerala & 11/2/2017 & Y & Eviction & Deferred  & The petitioner to show cause; no eviction until a final decision is taken & 1 \\

2018 SCC ONLINE MAD 1392 & N. Murugan v. Joint Commissioner  & WP(MD) 10584 of 2018 & High Court of Madras & Tamil Nadu & 5/3/2018 & Y & Fee-collection right to be auctioned & Dismissed & \quotes{*Public auction cum tender notification (right to collect fees from street vendors) for auctioning the collection right challenged

* Grounds for a challenge: a representation submitted for the renewal of his right, he being the successful bidder for the previous year

* Representation already rejected by the government

* Writ dismissed with liberty to challenge the order rejecting the request of petitioner.} & 2\\

2018 SCC ONLINE DEL 6813 & Nanni Bai Ahirwar v. New Delhi Municipal Council and Anr & WP(C) 673/2018 & High Court of Delhi & Delhi & 1/23/2018 & Y & Permission for the change of trade & Deferred  & \quotes{(1) The petitioner would approach the TVC as and when it is constituted with all the supporting documents.

(2) TVC will consider the case of the petitioner in accordance with law and expeditiously after considering all the material on record.

(3) Merely because the petitioner is not found vending at the site when the survey is conducted, that by itself would not be a ground alone to reject his case.} & 1 \\

2017 SCC ONLINE MAD 27698 & P. Arun Kumar v. State Commissioner for Differently Abled   & WP 19858 of 2017 & High Court of Madras & Tamil Nadu & 9/20/2017 & Y & Eviction; definition of street vendor & Deferred  & \quotes{* Quoted WPC 18677 of 2014, order dated 3.09.2015: "street vendor" includes one selling from a temporary built-up structure.

* TVC yet to be constituted: Respondent 2-4 shall consider the claim of the petitioner re: applicability of SVA 2014 within 12 weeks; until then the bunk shop not to be disturbed.} & 2\\

2017 SCC ONLINE MAD 2159 & P. Rajam v. Corporation of Chennai & WP 22515/2014 & High Court of Madras & Tamil Nadu & 6/7/2017 & Y & CoV & Deferred & The petitioner may move an application to TVC within 2 weeks. & 1\\

MANU/SCOR\\/01337/2017 & Paguthi Small Viyaparigal Sangam and State of Tamil Nadu and Ors & Petitions for Special Leave to Appeal C Nos. 857/2017 & Supreme Court of India & India & 1/11/2017 & Y & Eviction & Deferred  &  & 1\\

MANU/TN/55\\38/2018 & Palani v. Government of Tamil Nadu and Ors &WP 2232/2018 and WMP 2730/2018 & High Court of Madras & Tamil Nadu & 9/18/2018 & Y & Vending zone; relocation; implementation & Directions & Noted the steps for implementation already being taken by government authorities to implement the Act. & 2 \\

2017 SCC ONLINE MAD 30457 & Ramanathan Theru, Usman Road Kizhaku Paguthi Small Viyaparigal Sangam v. A.K. Vishwanathan & Cont. Pet. No. 952/2017 & High Court of Madras & Tamil Nadu & 10/4/2017 & Y & Contempt; eviction & Directions & Police not to cause any disturbance to existing street vendors until a decision is taken by the vending committee. Police prevented only new entrants so as to maintain the free flow of traffic and avoid any public inconvenience. & 3 \\

MANU/RH/02\\12/2017 & Ravindra Singh and Ors. v. State of Rajasthan and Ors & \quotes{SB CWP 17763, 8415/2016, 10451, 1026/2015, 377, 5019 and 4471/2017} & High Court of Rajasthan & Rajasthan & 4/6/2017 & Y & Eviction & Directions & \quotes{(Para 11) Till the directions aforesaid are completed, the existing vendors would not be disturbed in the light of Sub-section 3 of Section 3 of the Act of 2014; however, directions would not apply to the area which is otherwise covered by any other judgment of this Court directing the Municipal Corporation not to allow vending in that area.} & 3 \\

2017 SCC ONLINE KER 13854 & Sabu M.J v. Corporation of Kochi & WP(C)10547/2017 & High Court of Kerala & Kerala & 3/30/2017 & Y & Eviction & Deferred &Direction to the Deputy Mayor to consider the petitioner's representation in accordance with law within 2 weeks; no eviction without relocation. & 3 \\

MANU/KE/20\\97/2018 & Saji Joy v. The Executive Engineer, PWD NH Division and Ors & W.P.(C)18356, 23827/2018 & High Court of Kerala & Kerala & 8/7/2018 & Y & Eviction; relocation; legislative overlap & Deferred & Competent authority to consider applications after hearing the petitioners without much delay. & 2\\

MANU/UP/12\\73/2018 & Satender Kumar and Ors v. State of U.P. and Ors & Writ-C 8602/2018 & High Court of Allahabad & Uttar Pradesh & 3/7/2018 & Y & Eviction; implementation & Directions & \quotes{The authority shall undertake and complete the survey as contemplated under the 2014 Act and proceed to identify and create vending zones in accordance with the statutory provisions. As and when the vending zones are created, it shall be open for the petitioners to seek their registration as street vendors under the Act aforementioned and apply for an allotment.} & 3\\

MANU/SCOR\\/17149/2018 & Self-Employed Welfare Association v. The State Commissioner for Differently Abled \& Ors & SLP C 9857/2018 & Supreme Court of India & India & 5/18/2018 & Y & Survey; implementation & Directions & Survey to be conducted within a month. & 2 \\

2017 SCC ONLINE KER 12837 & Shamsu C v. Corporation of Cochi & WP(C) 25597/2013 & High Court of Kerala & Kerala & 3/24/2017 & Y & Eviction; relocation; entitlement & Deferred & For factual determination of the question of entitlement of the petitioner to occupy the bunk shop, an opportunity to be provided to the petitioner to put forth his claim before the corporation of Kochi or TVC if already constituted. & 3\\

2017 SCC ONLINE DEL 8202 & Sheetal Prasad Gupta v. New Delhi Municipal Council & WP(C)11592/2016 & High Court of Delhi & Delhi & 4/25/2017 & Y & Eviction; relocation; definition & Directions; consent decree & Name of the petitioners find mentioned in the list of 628 vendors prepared by the NDMC; hence to be relocated. & 2\\

2018 SCC ONLINE DEL 8150 & Sita Rawat v. South Delhi Municipal Corporation & WP(C) 1293/2018 & High Court of Delhi & Delhi & 3/7/2018 & Y & Eviction & Deferred & \quotes{(1) The petitioner would approach the TVC as and when it is constituted with all the supporting documents.

(2) TVC will consider the case of the petitioner in accordance with law and expeditiously after considering all the material on record.

(3) Merely because the petitioner is not found vending at the site when the survey is conducted, that by itself would not be a ground alone to reject his case.

(4) Rules are notified.} & 1\\

MANU/KE/17\\24/2018 & Sudheesh T.S v. State of Kerala and Ors & WP(C) 17792/2018 & High Court of Kerala & Kerala & 7/10/2018 & Y & Eviction; relocation; legislative overlap & Deferred & Competent authority to consider applications after hearing the petitioners without much delay. &  \\

2017 SCC ONLINE BOM 571 & Thane Zilla (Maharashtra) Hawkers Union v State of Maharastra and Ors & WP (ST) 4622/2017 & High Court of Bombay & Maharash tra & 3/6/2017 & Y & TVC constituted without street vendor representation & Directions & \quotes{1) Local authorities to comeback with a statement as to how they intent to remove the anamoly created;

2) Local authority to keep 8 vacancies unfilled in the TVC can proceed with the other 12 members.} & 3 \\

MANU/KE/17\\22/2018 & V. Prabhakaran and Ors v. National Highway Authority, Kozhikode and Ors & WP(C) 17377/2018 & High Court of Kerala & Kerala & 7/17/2018 & Y & Eviction; relocation; legislative overlap & Deferred & Competent authority to consider applications after hearing the petitioners without much delay. & 2\\

MANU/DE/20\\90/2018 & Vijay Kumar Sahu and Ors v. Govt. of NCT and Ors & WP(C) 7785/2017 & High Court of Delhi & Delhi & 5/29/2018 & Y & Eviction & Deferred & Rules notified; public notice issued for inviting applications with supporting documents; the petitioner to approach the TVC as and when it is functional; TVC to consider the case. & 2\\

2018 SCC ONLINE DEL 7578 & Vijender v. South Delhi Municipal Corporation & WP(C) 2291/2017 & High Court of Delhi & Delhi & 2/12/2018 & Y & Eviction & Deferred & Adjourned the matter for a period of 3 months to enable theTVC to become functional. & 2 \\

2017 SCC ONLINE DEL 11779 & Virender v. South Delhi Municipal Corporation & WP(C) 11552/2016 & High Court of Delhi & Delhi & 10/25/2017 & Y & Definition of street vendor; eviction & Partially dismissed; consent decree & No-vending zone as decided prior to the Act to be continued; the Act is applicable only to \quotes{regular street vendors}. No relief to be granted to petitioners 1 to 9 who are not regular street vendors. & 2 \\

\end{longtable}
\end{landscape}


%Appendix 2
\newpage
\section*{Appendix 2: Formula Used to Calculate the Index}
\addcontentsline{toc}{section}{Appendix 2: Formula Used to Calculate the Index}
\label{sec: Appendix 2}
Formula used to calculate the index:
\begin{align*}
SVC_n &= \sum_{i = 1}^{i = n} \alpha_i V_i
\end{align*}


The focus of the index is first on rule-making and establishment of institutions and bodies, for example, TVCs and Grievance Redressal Committees and only after on implementation steps. Without institutions, as mandated by the Act, implementation can be challenged.



%Table 7: Variables and Weights Used for Calculation of State Score
\begin{longtable}[l]{>{\raggedright}p{1.5cm}>{\raggedright}p{6cm}>{\raggedright}p{2.5cm}>{\raggedright\arraybackslash}p{4cm}}
  \caption{Variables and Weights Used for Calculation of State Score}\\
    \toprule
$V_i$ & Variable & Weight $(\alpha_i)$ & Variable into Weight \\
$V_i$ & Variable & Weight $(\alpha_i)$ & Variable into Weight \\
\midrule
\endhead
\bottomrule
\endfoot
\bottomrule
\endlastfoot
$V_1$ & Whether the state government has notified the rules for implementing the Act (1 if drafted; 0 if not drafted) & 13 & $13 * V_1$\\
$V_2$ & Whether the state government has notified the scheme for implementing the Act (1 if drafted; 0 if not drafted) & 13 & $13 * V_2$\\
$V_3$ & Proportion of towns with a Grievance Redressal Committee & 11 & $11 * V_3$\\
$V_4$ & Proportion of towns with TVC & 11 & $11 * V_4$\\
$V_5$ & Proportion of TVCs with \textbf{vendor representatives} & 10 & $10 * V_5$\\
$V_6$ & Proportion of TVCs that have \textbf{conducted survey} & 10 & $10 * V_6$\\
$V_7$ & Proportion of TVCs that have \textbf{issued identity cards to more than 75\% of identified vendors} & 8 & $8 * V_7$\\
$V_8$ & Proportion of TVCs that have \textbf{earmarked vending zones} & 7 & $7 * V_8$\\
$V_9$ & Proportion of TVCs that have \textbf{a vending plan} & 7 & $7 * V_9$\\
$V_{10}$ & Proportion of TVCs that have \textbf{published a charter} & 6 & $6 * V_{10}$\\
$V_{11}$ & Proportion of TVCs that have \textbf{assigned office space} & 4 & $4 * V_{11}$\\
\midrule
& & 100 & $\sum_{i = 1}^{i = n} \alpha_i V_i$\\
\end{longtable}

%Appendix 3
\newpage
\section*{Appendix 3: Questionnaire for SV Act Implementation}
\addcontentsline{toc}{section}{Appendix 3: Questionnaire for SV Act Implementation}
\label{sec: Appendix 5}


\begin{mdframed}[backgroundcolor=gray!20]
\textbf{Module A: Basic information}

A1. Name of the respondent:

A2. Gender: M/F/Other

A3. Designation in committee:


\textbf{Module B: Constitution of a Town Vending Committee}

B1. How many TVCs are there in Gurugram currently?

B2. When was this TVC formed? (mm/yyyy)

B3. How many members are there on the committee?

B4. What is your role in the committee? (Specify organisation where necessary)

\begin{enumerate}[nosep]
\item Municipal commissioner
\item Local body official
\item Traffic Police official
\item Local Police official
\item RWA member
\item Street vendors
\item Others, please specify
\end{enumerate}

Ask B5-6 if answer to B4 is Municipal commissioner

B5. Were the rules and scheme formed prior to the formation of the TVC? (ask for specific dates)

B6. Who is the Chairperson of the town vending committee of each ward/zone?

B7. Since when have you been a part of this TVC?

B8. How did you become a part of the committee?
\begin{enumerate}[nosep]
\item Nomination
\item Election
\item Others, please specify

B9. What are your responsibilities in the committee? (Enlist top five)

B10. What is your interest as a part of this committee? (Pick top three concerns)
\begin{enumerate}[nosep]
\item Voice resident concerns
\item Voice vendor interests and challenges
\item Represent government interests
\item Voice concerns of existing natural markets
\item Reduce street vendors in the area
\item Increase street vendors in the area
\item Ensure cleanliness
\item Prevent traffic congestion
\item To attend the meetings
\item Others, please specify
\end{enumerate}

\textbf{Module C: Operation and Activities}


C2. Where and how are minutes of it recorded?

C3. Is attendance marked? (Ask for records of the last 3 dates of meetings and MOM)

C4. Is there a designated office space for the Town Vending Committee? (Skip C3 if the answer is no)

C5. Where is it located?

C6. According to the rules, what activities have been completed by your TVC? (Use options to give examples)
\begin{enumerate}[nosep]
\item Election for electing street vendors
\item Formed Grievance Redressal Committee
\item Identified street vendors
\item Identified vending zones
\item Issued ID cards and certificate of vending
\item Drafted plan of vocation for street vendors
\end{enumerate}

C7. What is the order followed to complete the above activities?

C8: Has the committee marked vending zones? (Skip C5 if the answer is no)

C9. What factors are considered to mark vending zones? (Use options to give examples)
(By natural markets, we mean: a market where sellers and buyers have traditionally congregated for the sale and purchase of products or services)
\begin{enumerate}[nosep]
\item Traffic
\item Locations preferred by the public
\item Existing residents
\item Existing natural markets
\item Preference of vendors
\item Others:
\end{enumerate}


C11. Which body looks at vendor grievances?
\begin{enumerate}[nosep]
\item Grievance redressal committee
\item TVC
\item Local authority
\item Others
\end{enumerate}

C12. What is the process for filing grievances?

C13.  For which of the following does the committee maintain records?
\begin{enumerate}[nosep]
\item Total numbers of street vendors identified
\item Vending zones
\item Registered vendors
\item Licenses issued
\item Location of vendors
\item Revenue generated through fee collected
\item Challan issued
\item Fine imposed
\item No, we do not maintain records
\end{enumerate}

C14. How are the records maintained? (Request documents)
\begin{enumerate}[nosep]
\item Electronically
\item Manually
\end{enumerate}


C15. What are the sources of revenue for the committee?
\begin{enumerate}[nosep]
\item Funded by the government
\item Vending fee
\item Challans/fines
\item Others
\end{enumerate}


C16. What is the amount for the following charges collected from the vendors:
\begin{enumerate}[nosep]
\item Vending fee
\item Maintenance fee
\item Others
\end{enumerate}


C17. Is the vending/maintenance fee uniform in all vending zones?


C18. What is the procedure in case a vendor can’t pay the vending/maintenance fee?


C19. What fines/challans are applicable to the vendors, in case of non-compliance:
\begin{enumerate}[nosep]
\item Penalty fee
\item Challans
\item Others
\end{enumerate}

\textbf{Module D: Street Vendors’ Representation and Election}

(The Act mandates 40\% of any TVC to be street vendors with due representation to SC, ST and other backward classes)

D1. Does the TVC have street vendors’ representatives? If no, why not?

Ask D2-D8 only if the answer to D1 is yes

D2. How many street vendors’ representatives does the TVC have?

D3. If yes, how were the members appointed?
\begin{enumerate}[nosep]
\item Nomination
\item Election
\item Lottery system
\item Other, please specify
\end{enumerate}

Ask D4-D9 only if the answer to D2 is ‘election’

D4. How were the elections announced?

D5. Which authority/officer was responsible to conduct the elections?
\begin{enumerate}[nosep]
\item Municipal Corporation
\item Municipal Council/Committee
\item Planning Authority
\item Other, please specify
\end{enumerate}


D7. How are the zones decided for elections?

       zones and vendors left out)
D8. What is the basis of shortlisting the contesting candidates?
\begin{enumerate}[nosep]
\item Compliance of mandatory provisions as under Act and Rules
\item Missing signature on declaration form for nomination
\item Other, please specify
\end{enumerate}

D9. How are the names of elected members notified?
\begin{enumerate}[nosep]
\item Website
\item ULB office
\item Other, please specify
\end{enumerate}
\end{mdframed}

\begin{mdframed}[backgroundcolor=gray!20]
\subsection*{Private Enterprise hired for Act Implementation under Pilot Project}

\textbf{Module A: Basic information}
A1. Name of the respondent:

A2. Organization:

A3. Designation:

A4. What does your enterprise do?

\textbf{Module B: Their role under SV Project}

B1. How was your agency selected for this project?


B3. Which zones are your responsible for? (Enlist all)


B5. What is the amount for the following charges collected from the vendors:
\begin{enumerate}[nosep]
\item Convenience fee
\item Cart fee
\item Maintenance fee
\item Others
\end{enumerate}

B6.  What is the mode of fee collection?

B7. What is the process followed for fee collection?

B8. How are revenues shared with the MCG?

B9. Do you assist the MCG in registration of vendors?


B11. Do you distribute any IDs to the vendors? (Share copy)

B12. Do you have a role in spatial mapping of zones under the project?


B14. What factors are considered to inspect vendors?
\begin{enumerate}[nosep]
\item Adulteration (FSSAI norms), in case of food vendors
\item Others
\end{enumerate}

B15. Does any team from MCG visit the sites under your jurisdiction?


B17.  For which of the following does your agency maintain records?
\begin{enumerate}[nosep]
\item Total numbers of street vendors identified
\item Vending zones
\item Registered vendors
\item ID cards issued
\item Revenue generated through fee collected
\item Others
\end{enumerate}


\textbf{Module C: Town Vending Committee}

C1. Are you aware of the Street Vendors Act 2014?

C2. Are you aware about the Town Vending Committee (TVC)?

C3. Are you a part of the TVC? (Skip B4-B16 if the answer is no)

C5. What is your role in the committee? (Specify organisation where necessary)

C6. Since when have you been a part of this TVC?

C7. How many meetings of the TVC were held in the last 6 months?

C8. Does the TVC have street vendors’ representatives? (Skip B15 if the answer is no)

C9. How many street vendors’ representatives does the TVC have?

C10. Which body looks at vendor grievances?
\begin{enumerate}[nosep]
\item Grievance redressal committee
\item TVC
\item Local authority
\item Your agency
\item Others
\end{enumerate}

C12. What is the process for filing grievances?
\end{mdframed}

\begin{mdframed}[backgroundcolor=gray!20]
\subsection*{Street Vendor}
\textbf{Module A: Basic information}

A2. Gender:

A3. Type of vendor:
\begin{enumerate}[nosep]
\item Mobile/Street
\item Fixed
\end{enumerate}

A4. Area of vending:


A6. Are you a part of any association?
\begin{enumerate}[nosep]
\item Yes
\item No
\end{enumerate}
\textbf{Module B: Income and rents}
B1. For how long have you been vending:
\begin{enumerate}[nosep]
\item 0-5 years
\item 5-10 years
\item 5-15 years
\item 15+ years
\end{enumerate}

B2. What is your daily revenue/total sales?
\begin{enumerate}[nosep]
\item $>$300
\item 500-1000
\item 1000-2000
\item 2000-3000
\item 3000+
\end{enumerate}

B3. What is your profit per day?
\begin{enumerate}[nosep]
\item $>$300
\item 500-1000
\item 1000-2000
\item 2000-3000
\item 3000+
\end{enumerate}

B4. Do you pay a vending fee to the committee?

\textbf{Yes}

\textbf{No}

B5. Are you a beneficiary of any government scheme? If yes, please specify the name of the scheme.
\textbf{Module C: Awareness and changes since implementation of the Act}
C1. Are you aware of the Street Vendors Act 2014?

\textbf{Yes}

\textbf{No}

C2. What has changed in the past one year?
\begin{enumerate}[nosep]
\item I face less harassment by the police/other officials.
\item I do not have to pay bribes/payment in kind.
\item The government has allocated a vending space to me.
\item I have a new cart.
\item No change
\item Others
\end{enumerate}

C3. Is there a Town Vending Committee in your zone?

\textbf{Yes}

\textbf{No}

\textbf{Don’t know}

C4. Do you have a license/certificate of vending? (If yes, ask for a copy)

\textbf{Yes}

\textbf{No, I am unregistered}

\textbf{Module D: Harassment/exploitation by police and other authorities}

D1. Do you have visits by the following?
\begin{enumerate}[nosep]
\item Government/ULB official
\item Police officer
\item Shopkeeper Association
\item Others
\end{enumerate}

Ask D2-D5 if the answer to D1 is yes

D2. What is the frequency by each of the above?
\begin{enumerate}[nosep]
\item Daily
\item Weekly
\item Fortnightly
\item Monthly
\item Once in two months
\item Quarterly
\item Bi-annually
\item Annually
\end{enumerate}

D3. What is the intent of the inspectors/officials? (You may select more than one option)
\begin{enumerate}[nosep]
\item To ease the conduct of your business
\item To check for violations
\item To ensure the welfare of public
\item To address traffic congestion
\item To ensure cleanliness \& public hygiene
\item To collect fees/charges
\item To avail free service
\item To harass and get bribes or benefits
\end{enumerate}

D4. What are you expected to do in case they find faults during these visits? (You may select more than one option)
\begin{enumerate}[nosep]
\item Relocate to another area: (with/without notice)
\item Get your goods seized by officials
\item Pay a bribe
\end{enumerate}

D5. Where do you voice your problems?
\begin{enumerate}[nosep]
\item TVC
\item Street vendor associations
\item Grievance redressal committee
\item Vendor leader
\item Others
\end{enumerate}

\textbf{Module E: Street Vendors’ Representation and Election (if answer to B3 is yes))}

E1. Does the TVC have street vendors’ representatives?

E2. If yes, how were the members appointed?
\begin{enumerate}[nosep]
\item Nomination
\item Election
\item Other, please specify
\end{enumerate}

(Ask E3-E5 only if the answer to E2 is b)


E4. How were the elections announced?

       (to check for information asymmetry to vendors)
E5. Who conducted the elections?




%Appendix 4
\newpage
\section*{Appendix 4: List of TVC members as mandated by MCG}
\label{sec: Appendix 6}
\addcontentsline{toc}{section}{Appendix 4: List of TVC members as mandated by MCG}
\begin{figure}[h]
\centering
\includegraphics[width=5.1in]{Appendix6.pdf}
\end{figure}

%Appendix 5
\newpage
\section*{Appendix 5: List of TVC members as per TVC Meeting of August 2018}
\addcontentsline{toc}{section}{Appendix 5: List of TVC members as per TVC Meeting of August 2018}
\begin{figure}[h]
\centering
\includegraphics[width=7in]{ListofTVC.pdf}
\end{figure}

%Appendix 6
\newpage
\section*{Appendix 6: Receipt of Maintenance Fee Paid by \\Vendors}
\addcontentsline{toc}{section}{Appendix 6: Receipt of Maintenance Fee Paid by Vendors}
\begin{figure}[h]
\centering
\includegraphics[width=6in]{Receipt1.pdf}
\end{figure}

%Appendix 7
\newpage
\section*{Appendix 7: Complaint Made By a Private Party to MCG against RWA}
\begin{figure}[h]
\centering
\includegraphics[height=5.5in]{Receipt2.pdf}
\end{figure}
%====================================================================
